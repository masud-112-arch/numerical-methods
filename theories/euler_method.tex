\documentclass[11pt,a4paper]{article}
\usepackage[utf8]{inputenc}
\usepackage{amsmath,amssymb}
\usepackage{geometry}
\usepackage{graphicx}
\geometry{margin=1in}

\title{Euler Method: Explanation and Worked Example}
\author{Ibrahim Hossen Masud}
\date{\today}

\begin{document}
\maketitle

\begin{abstract}
This note presents the Euler method for solving ordinary differential equations (ODEs). 
It includes the derivation of the method, a step-by-step worked example, and discussion on accuracy and limitations.
\end{abstract}

\section{Introduction}
The Euler method is one of the simplest numerical techniques for approximating solutions of ordinary differential equations of the form:
\[
\frac{dy}{dt} = f(t,y), \quad y(t_0) = y_0
\]
It provides a first-order approximation, meaning that the error decreases linearly with the step size.

\section{Derivation of Euler Method}
Starting from the Taylor expansion of $y(t)$ around $t_n$:
\[
y(t_{n+1}) = y(t_n + h) = y(t_n) + h \frac{dy}{dt}\bigg|_{t_n} + \frac{h^2}{2} \frac{d^2y}{dt^2}\bigg|_{\xi_n}, \quad \xi_n \in [t_n, t_{n+1}]
\]
Neglecting the second-order term gives:
\[
y_{n+1} \approx y_n + h f(t_n, y_n)
\]
which is the Euler update formula.

\section{Worked Example}
Consider the differential equation:
\[
\frac{dy}{dt} = -2y + 2 - e^{-4t}, \quad y(0) = 1
\]
with exact solution:
\[
y(t) = 1 + \frac{1}{2} e^{-4t} - \frac{1}{2} e^{-2t}
\]

We choose a step size \(h = 0.1\) and compute the first few iterations:

\[
\begin{aligned}
y_1 &= y_0 + h f(t_0, y_0) = 1 + 0.1(-2\cdot 1 + 2 - e^{0}) = 1 - 0.1 = 0.9 \\
y_2 &= y_1 + h f(t_1, y_1) = 0.9 + 0.1(-2\cdot 0.9 + 2 - e^{-0.4}) \approx 0.9 + 0.1(-1.8 + 2 - 0.6703) \\
&\approx 0.9 + 0.1(-0.4703) \approx 0.85297
\end{aligned}
\]

Continuing this process, one can approximate \(y(t)\) for any \(t\) in the interval of interest.

\section{Discussion on Accuracy and Limitations}
\begin{itemize}
    \item Euler method is first-order accurate, with global truncation error proportional to the step size \(h\).  
    \item For small step sizes, the method gives reasonable approximations; however, for stiff equations or large \(h\), it may become unstable.  
    \item It serves as a foundation for more advanced methods such as Heun's method, Runge-Kutta methods, and multistep methods.  
    \item The method is easy to understand and implement, making it ideal for introductory study and teaching.
\end{itemize}

\section{Conclusion}
Euler method provides a simple way to numerically solve ODEs. While limited in accuracy, its conceptual simplicity makes it a crucial stepping stone toward more sophisticated numerical schemes.

\end{document}