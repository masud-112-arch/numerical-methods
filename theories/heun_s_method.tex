\documentclass[11pt,a4paper]{article}
\usepackage[utf8]{inputenc}
\usepackage{amsmath,amssymb}
\usepackage{geometry}
\usepackage{graphicx}
\geometry{margin=1in}

\title{Heun's Method: Explanation and Worked Example}
\author{Ibrahim Hossen Masud}
\date{\today}

\begin{document}
\maketitle

\begin{abstract}
This note presents Heun's method for solving ordinary differential equations (ODEs). 
It includes the derivation of the method, a step-by-step worked example, and discussion on accuracy and advantages over Euler's method.
\end{abstract}

\section{Introduction}
Heun's method is a predictor-corrector method that improves the accuracy of Euler's method. It is a second-order method for solving initial value problems of the form:
\[
\frac{dy}{dt} = f(t,y), \quad y(t_0) = y_0
\]

\section{Derivation of Heun's Method}
Heun's method proceeds in two steps:

\subsection{Predictor Step (Euler Estimate)}
\[
y_{n}^{p} = y_n + h f(t_n, y_n)
\]
This is the Euler estimate at the next step.

\subsection{Corrector Step (Average Slope)}
\[
y_{n+1} = y_n + \frac{h}{2} \Big[f(t_n, y_n) + f(t_{n+1}, y_n^{p}) \Big]
\]

\textbf{Explanation:} The method approximates the solution by taking the average of the slope at the beginning and the predicted slope at the end of the interval, giving a second-order accuracy.

\section{Worked Example}
Consider the ODE:
\[
\frac{dy}{dt} = y - t^2 + 1, \quad y(0) = 0.5
\]
with exact solution:
\[
y(t) = (t+1)^2 - 0.5 e^t
\]

Choose step size \(h = 0.2\).

\subsection{Step 1}
\[
\begin{aligned}
\text{Predictor: } y_0^p &= y_0 + h f(t_0, y_0) = 0.5 + 0.2 \big(0.5 - 0^2 + 1\big) = 0.5 + 0.2(1.5) = 0.8 \\
\text{Corrector: } y_1 &= y_0 + \frac{h}{2}\big[f(t_0, y_0) + f(t_1, y_0^p)\big] \\
&= 0.5 + 0.1\big[1.5 + (0.8 - 0.2^2 + 1)\big] \\
&= 0.5 + 0.1(1.5 + 1.56) = 0.5 + 0.306 = 0.806
\end{aligned}
\]

\subsection{Step 2 (Optional)}
\[
\begin{aligned}
y_1^p &= y_1 + h f(t_1, y_1) = 0.806 + 0.2 (0.806 - 0.2^2 + 1) \approx 1.170 \\
y_2 &= y_1 + \frac{h}{2} [f(t_1, y_1) + f(t_2, y_1^p)] \approx 0.806 + 0.1(1.566 + 1.930) \approx 1.1596
\end{aligned}
\]

Continue similarly to approximate \(y(t)\) for the desired interval.

\section{Discussion on Accuracy and Advantages}
\begin{itemize}
    \item Heun's method is second-order accurate, with global error proportional to \(h^2\), compared to Euler's first-order accuracy.  
    \item Using the average slope reduces truncation error and improves stability.  
    \item Slightly more computational work than Euler method (one extra function evaluation per step), but much higher accuracy.  
    \item Provides a simple introduction to predictor-corrector and Runge-Kutta methods.
\end{itemize}

\section{Conclusion}
Heun's method improves upon Euler by averaging slopes at the beginning and end of each step. It is simple, accurate, and forms the basis for more advanced numerical ODE techniques.

\end{document}