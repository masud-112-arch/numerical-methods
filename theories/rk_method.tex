\documentclass[11pt,a4paper]{article}
\usepackage[utf8]{inputenc}
\usepackage{amsmath,amssymb}
\usepackage{geometry}
\usepackage{graphicx}
\geometry{margin=1in}

\title{Runge-Kutta 4th Order Method: Explanation and Worked Example}
\author{Ibrahim Hossen Masud}
\date{\today}

\begin{document}
\maketitle

\begin{abstract}
This note presents the classical 4th-order Runge-Kutta (RK4) method for solving ordinary differential equations (ODEs). 
It includes the derivation, a worked example, and discussion on accuracy and stability.
\end{abstract}

\section{Introduction}
The Runge-Kutta methods are a family of iterative techniques for solving initial value problems:
\[
\frac{dy}{dt} = f(t,y), \quad y(t_0) = y_0
\]

The 4th-order Runge-Kutta (RK4) method is one of the most commonly used due to its high accuracy and simplicity for single-step methods. It is fourth-order accurate, meaning the global truncation error scales with \(h^4\).

\section{Derivation of RK4 Method}
The RK4 method computes intermediate slopes to approximate the solution:

\[
\begin{aligned}
k_1 &= f(t_n, y_n) \\
k_2 &= f\left(t_n + \frac{h}{2}, y_n + \frac{h}{2} k_1\right) \\
k_3 &= f\left(t_n + \frac{h}{2}, y_n + \frac{h}{2} k_2\right) \\
k_4 &= f(t_n + h, y_n + h k_3)
\end{aligned}
\]

The next step is then computed as a weighted average of these slopes:
\[
y_{n+1} = y_n + \frac{h}{6} \left(k_1 + 2k_2 + 2k_3 + k_4\right)
\]

\textbf{Explanation:}  
- \(k_1\) is the slope at the beginning of the interval  
- \(k_2\) and \(k_3\) are slopes at the midpoint  
- \(k_4\) is the slope at the end  
- The weighted average gives a fourth-order accurate approximation

\section{Worked Example}
Consider the differential equation:
\[
\frac{dy}{dt} = t + y, \quad y(0) = 1
\]
with exact solution:
\[
y(t) = -t - 1 + 2 e^t
\]

Choose step size \(h = 0.1\).

\subsection{Step 1}
\[
\begin{aligned}
k_1 &= f(t_0, y_0) = 0 + 1 = 1 \\
k_2 &= f(t_0 + 0.05, y_0 + 0.05 \cdot 1) = f(0.05, 1.05) = 0.05 + 1.05 = 1.10 \\
k_3 &= f(t_0 + 0.05, y_0 + 0.05 \cdot 1.10) = f(0.05, 1.055) = 0.05 + 1.055 = 1.105 \\
k_4 &= f(t_0 + 0.1, y_0 + 0.1 \cdot 1.105) = f(0.1, 1.1105) = 0.1 + 1.1105 = 1.2105 \\
y_1 &= y_0 + \frac{0.1}{6} (1 + 2 \cdot 1.10 + 2 \cdot 1.105 + 1.2105) \\
&= 1 + 0.016667 (1 + 2.2 + 2.21 + 1.2105) \\
&= 1 + 0.016667 \cdot 6.6205 \approx 1 + 0.1103 = 1.1103
\end{aligned}
\]

\subsection{Step 2 (Optional)}
Repeat similarly to compute \(y_2, y_3, \dots\) for the interval of interest.

\section{Discussion on Accuracy and Advantages}
\begin{itemize}
    \item RK4 is fourth-order accurate, much more precise than Euler (first-order) or Heun (second-order) for the same step size.  
    \item It is a single-step method, requiring no previous history beyond the last step.  
    \item More computationally intensive than Euler or Heun (4 slope evaluations per step), but highly stable and reliable for general ODEs.  
    \item Provides a standard reference method for testing and benchmarking other numerical solvers.
\end{itemize}

\section{Conclusion}
The classical RK4 method is a robust and accurate numerical technique for solving ODEs. Its balance of accuracy, stability, and simplicity makes it widely used in scientific computing.

\end{document}